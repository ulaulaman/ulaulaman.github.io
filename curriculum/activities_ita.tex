\documentclass[a4paper,latin1,italian]{article}

\usepackage{tabularx}

%\usepackage{doublespace}
%\setstretch{1.2}

\usepackage[latin1,utf8x]{inputenc}
\usepackage[italian]{babel}
\usepackage{ae}
\usepackage[T1]{fontenc}
\usepackage{CV}

\usepackage{ifpdf}
\ifpdf
\usepackage[hyperindex]{hyperref}
\pdfadjustspacing=1
\fi
%
\usepackage{hyperref}
 \hypersetup{
  pdfpagemode=UseOutlines,
  %pdfstartview=FitV,
  %bookmarksopen,
  %bookmarksopenlevel=-1,
  pdftitle=Curriculum Vitae,
  pdfauthor=Gianluigi Filippelli,
  %pdfsubject=,
  pdfkeywords=LaTeX
  %pdfpagemode=FullScreen
  }

\begin{document}

\pagestyle{empty}

\begin{center}
\huge{\textsc{Lista articoli e descrizione attivit\`a}}
\vspace{\baselineskip}
	
	\Large{\textsc{Gianluigi Filippelli}}
\end{center}
\vspace{1.5\baselineskip}

\section{Pubblicazioni}
\begin{itemize}
	\item Ph.D thesis: {\em Rappresentazioni proiettive: Applicazioni alla teoria quantistica}, 2006, Universit\`a della Calabria
	\item Filippelli G. {\em Science Backstage} Scienza Aperta Vol. III (2012) - ComunicareFisica2010, Atti $3^\circ$ Convegno "Comunicare Fisica e altre Scienze", Frascati 12-16 aprile 2010, \href{http://www.lnf.infn.it/sis/frascatiseries/italiancollection/index_ita.php}{Frascati Physics Series - Italian Collection}
	\item Filippelli G., {\em Time dependent quantum generators for the Galilei group} Journal of Mathematical Physics, 52 (8) DOI: \href{http://dx.doi.org/10.1063/1.3621518}{10.1063/1.3621518}
	\item Filippelli G., Colaiacovo M., {\em United we stand. Siamo pronti per fare network?} Scienza Aperta Vol. IV (2014) - ComunicareFisica2012, Atti $4^\circ$ Convegno "Comunicare Fisica e altre Scienze", Torino 8-12 ottobre 2012, \href{http://www.lnf.infn.it/sis/frascatiseries/italiancollection/index_ita.php}{Frascati Physics Series - Italian Collection}
	\item Filippelli G., {\em Moodle for Italian Astronomy Olimpiad} DOI: \href{http://dx.doi.org/10.5281/zenodo.6761}{10.5281/zenodo.6761}
	\item articolo 1 giovanni
	\item articolo 2 giovanni
	\item report eduinaf
\end{itemize}
%
\section{Festival e altri eventi di divulgazione}
\begin{itemize}
	\item {\bf Focus Live 2018}: attivit� presso lo stand INAF con il videogioco di esplorazione spaziale Kerbal Space Program
	\item {\bf Play Modena 2019}: attivit� presso lo stand INAF con il videogioco di esplorazione spaziale Kerbal Space Program
	\item {\bf Futuro Remoto 2020}: operatore INAF Online Lab per il laboratorio A caccia di spettri
	\item {\bf National Geographic Festival della Scienza 2020}: operatore INAF Online Lab per il laboratorio Il Sistema Solare sulla mappa della tua citta'
\end{itemize}
%
\section{Attivit� presso l'Osservatorio Astronomico di Brera}
\begin{itemize}
	\item Serie di conferenze "La fisica con i supereroi"
	\item Gestione dei progetti di Alternanza Scuola-Lavoro prima e PCTO poi legati al videogioco di esplorazione spaziale Kerbal Space Program e al gaming in generale
	\item Corsi di formazione per insegnanti per l'uso in classe del videogioco di esplorazione spaziale Kerbal Space Program presso le sedi dell'INAF di Milano, Napoli e Bologna
	\item Gestione di un evento di presentazione con il videogioco di esplorazione spaziale Kerbal Space Program in occasione della Milano Digital Week 2019
	\item Formazione per le guide della mostra Walking on the Moon realizzata dall'Universit� Bicocca di Milano e assistenza per la giornata inaugurale della mostra relativamente all'attivit� legata al videogioco di esplorazione spaziale Kerbal Space Program
	\item Scrittura di articoli e realizzazione di video per il magazine MediaINAF\footnote{inserire permalink}
	\item Serie di conferenze sulla storia della fisica per docenti presso l'Istituto "G. Cavalleri" di Parabiago
	\item Supporto per attivit� di didattica e divulgazione (assistenza nella produzione dei video del progetto "Luna50")
	\item Supervisione attivit� di stagisti presso l'Osservatorio Astronomico di Brera (articoli per EduINAF; revisione di schede didattiche da pubblicare su EduINAF; progettazione di una escape room sulla storia dell'Osservatorio)
\end{itemize}
%
\section{Conferenze}
\begin{itemize}
	\item {\bf Padova 2017}: relatore al congresso SAIt con la presentazione del primo restyling di EduINAF
	\item {\bf Torino 2016}: relatore al workshop Comunicare la Chimica nell'epoca del web 2.0: \href{http://dropseaofulaula.blogspot.it/2016/09/web-20-da-fruitori-generatori-di.html}{Web 2.0: Da fruitori a generatori di informazione}
	\item {\bf Torino 2012}: relatore a ComunicareFisica. {\em United we stand. Siamo pronti per fare network?}: esame qualitativo e quantitativo dell'attivit\`a dei blogger italiani su Research Blogging. Relazione redatta con M. Colaiacovo
	\item {\bf Lucca 2012}: relatore a Lucca Comics\&Science: conferenza inaugurale dell'evento
	\item {\bf Frascati 2010}: relatore a ComunicareFisica. {\em Science Backstage}: presentazione dell'attivit\`a di blogger scientifico e divulgatore on-line
	\item {\bf Milano 2009}: relatore al Wordcamp. {\em Science Backstage}: presentazione dell'attivit\`a di blogger scientifico
	\item {\bf Prague 2007}: relatore al 16th Integrable systems and quantum symmetries. {\em Time dependent quantum generators for the Galilei group}: presentazione dei risulati sulla generalizzazione del gruppo di Galileo
	\item {\bf Bari 2004}: 12nd Workshop in Mechanical Statistics and Non Perturbative Fields Theory
	\item {\bf Cortona 2004}: 26th Informal Workshop in theoretical physics
\end{itemize}
\subsubsection*{Scuole di dottorato}
\begin{itemize}
	\item {\bf Otranto 2005}: Scuola Nazionale di Otranto: XVIII Seminario Nazionale di Fisica Nucleare e Subnucleare
	\item {\bf Otranto 2003}: Scuola Nazionale di Otranto: XVI Seminario Nazionale di Fisica Nucleare e Subnucleare
\end{itemize}
%
\newpage
%
\section{Attivit\`a per EduINAF}
%
\section{Olimpiadi Italiane dell'Astronomia}
Le Olimpiadi Italiane dell'Astronomia sono organizzate dalla Societ\`a Italiana di Astronomia in collaborazione con l'Istituto Nazionale di Astrofisica (INAF) e sono incluse nel programma del MIUR per l'educazione d'eccellenza. La Presidenza del Comitato Olimpico Nazionale ha sede presso l'Osservatorio Astronomico di Brera (INAF). Le Olimpiadi sono aperte agli studenti delle scuole superiori italiane: i vincitori delle prove nazionali partecipano alle Olimpiadi Internazionali dell'Astronomia (IAO). Le IAO nascono ufficialmente nel 1996 per iniziativa della Societ\`a Astronomica Euro-Asiatica. Sono organizzate ogni anno in autunno, in un paese differente, e vedono la partecipazione regolare di oltre venti squadre nazionali provenienti dall'area Europea e Asiatica, inclusa l'Italia. A tutt'oggi il supporto didattico \`e limitato al materiale fornito dal Comitato Italiano sul sito ufficiale italiano. Per colmare questa lacuna, ho progettato una piattaforma didattica, sviluppata con Moodle, che propone agli studenti delle pagine astronomiche nel formato di voci di glossario, una serie di esercizi, alcuni dei quali sviluppati con eXe-Learning, e alcune applett realizzate con geogebra.\\
(vedi {\em paper} allegato)

\section{Attivit\`a di blogging}
Dopo aver redatto varie voci su Wikipedia, molte delle quali su fisica e matematica, ho iniziato la mia attivit\`a di blogging nel gennaio 2009 con una collaborazione con il network italiano Blogosfere per un blog scientifico divulgativo (\emph{Science Backstage}). Questa esperienza ha permesso all'inaugurazione, insieme con altri blogger del campo, della sezione italiana dell'aggrgatore scientifico \href{http://researchblogging.org/}{Research Blogging}.\\
Con il successivo cambio editoriale in Blogosfere e la chiusura del blog, l'attivit\`a di divulgazione scientifica si \`e trasferita su due altri blog, \href{http://dropseaofulaula.blogspot.it/}{uno in italiano} e \href{http://docmadhattan.fieldofscience.com/}{l'altro in inglese}, quest'ultimo con il network scientifico {\em Field of Science}. Tra gli argomenti coperti vorrei enfatizzare una serie di post sulla didattica: alcuni esempi sono riportati nella nota\footnote{\href{http://dropseaofulaula.blogspot.it/2012/09/studio-in-classe-delle-biotracce-di-un.html}{Studio in classe delle biotracce di un pianeta extrasolare}, \href{http://dropseaofulaula.blogspot.it/2011/09/simulando-la-legge-di-hubble.html}{Simulando la legge di Hubble}}.

\section{Tesi di dottorato}
Nella tesi di dottorato, \emph{Rappresentazioni proiettive: Applicazioni nella teoria quantistica}, ho realizzato una trattazione unitaria della teoria delle rappresentazioni proiettive, per lo sviluppo di applicazioni nella teoria quantistica.\\
In particolare, ho trattato la teoria di Bargmann (1954), aggiungendo i contributi di Grigore (1996), Simms (1968), Varadarajan (1970) relativamente alla determinazione delle rappresentazioni proiettive di un dato gruppo di Lie, e il contributo di Karpilovsky (1994) sui caratteri delle rappresentazioni proiettive.\\
Questi contributi, insieme con i risultati di Bargmann sui gruppo abeliani e sui gruppi di trasformazioni pseudo-ortogonali, sono utilizzati nello studio dei gruppi di rotazione e di Galileo.\\
Nel primo caso, in accordo con Wigner (1939), Hurley (1966) e Altmann (1979), ho presentato l'emergere dello spin semi-intero, mentre nel secondo caso ho studiato il gruppo di Galileo prima in 3+1 dimensioni (Bargmann, 1954), quindi in (2+1) dimensioni (grigore, 1995; Bose, 1995) e infine in (1+1) dimensioni (Doebner e Mann, 1995).\\
In quest'ultimo caso, Doebner e Mann hanno calcolato l'esponente di fase del gruppo di Galileo con tre parametri e determinato i generatori infinitesimi dipendenti dal tempo. Ho esteso la procedura di Doebner-Mann a (2+1) e (3+1) dimensioni, calcolando una rappresentazione proiettiva di Galileo esplicitamente dipendente dal tempo.\\
Tali risultati sono stati presentati alla \emph{Integrable Systems' Prague Conference} del 2007 con la presentazione \emph{Time dependent quantum generators for the Galilei group} e in un articolo pubblicato sul \emph{Journal of Mathematical Physics}.
%
\section{Tesi di laurea}
Nella tesi di laurea, \emph{Produzione esaltata di coppie $e^+ e^-$ nelle collisioni tra ioni pesanti relativistici}, ho calcolato la sezione d'urto teorica della collisione tra ioni pesanti relativistici con l'obiettivo di spiegare la produzione di coppie $e^+ e^-$ utilizzando la teoria di Preparata (vedi G.Preparata, \emph{QED Coherence in Matter}, World Scientific, 1995). Ho determinato che i risultati ottenuti a partire dalla teoria di Preparata sono in disaccordo con i dati sperimentali (vedi R.J.Porter, S.Beedoe, R.Bossingham et al. \emph{Dielectron Cross Section Measurements in Nucleus-Nucleus Reactions at 1.0A GeV}, Phys. Rev. Lett. \textbf{79}, 1997).
\subsection*{Pagina web sul leptone $\mu$}
Nel corso di laurea, ho realizzato una pagina web sulla misura del tempo di vita media del leptone $\mu$ durante il corso "Laboratorio di fisica delle particelle elementari" utilizzando il blocco note. Il sito, cancellato dai server del Dipartimenti di Fisica dell'UNICAL, \`e accessibile su \href{http://web.archive.org/web/20070207064033/http://www.fis.unical.it/gruppi/alteenergie/}{Wayback Machine}.\\\\
%
\vspace{2\baselineskip}
\noindent Milano, \today

\end{document}