\documentclass[a4paper]{article}

\usepackage{tabularx}

%\usepackage{doublespace}
%\setstretch{1.2}

\usepackage[latin1]{inputenc}
%\usepackage[italian]{babel}
\usepackage{ae}
\usepackage[T1]{fontenc}
\usepackage{CV}

\usepackage{ifpdf}
\ifpdf
\usepackage[hyperindex]{hyperref}
\pdfadjustspacing=1
\fi
%
\usepackage{hyperref}
 \hypersetup{
  pdfpagemode=UseOutlines,
  %pdfstartview=FitV,
  %bookmarksopen,
  %bookmarksopenlevel=-1,
  pdftitle=Curriculum Vitae,
  pdfauthor=Gianluigi Filippelli,
  %pdfsubject=,
  pdfkeywords=LaTeX
  %pdfpagemode=FullScreen
  }

\begin{document}

\pagestyle{empty}

\begin{center}
	\huge{\textsc{Activities and papers}}
	\vspace{\baselineskip}
	
	\Large{\textsc{Gianluigi Filippelli}}
\end{center}
\vspace{1.5\baselineskip}

\section{Italian Astronomy Olympiads}
Italian Astronomy Olympiads are organized by Societ\`a Italiana di Astronomia in collaboration with Istituto Nazionale di Astrofisica (INAF) and they are included in the MIUR's upgrading program for educational excellence. The Presidency of the National Olympic Committee based at Osservatorio Astronomico di Brera (INAF). Olympiads are aimed to italian high school students: the winners of the national stage participate to International Astronomy Olympiad (IAO). IAO officially born in 1996 at the initiative of Euro-Asian Astronomical Society. They are held every year in autumn, in a different country, and they see the regular participation of over twenty national teams of the European and Asian area, including Italy. Today the didactical support is limited to the little material provided by Italian Committee on the italian official site. In order to supply this lack, I projected a didatcical platform, developed with Moodle, that it could propose astronomical learning pages like glossary's voices and a lot of exercises about every subject.

\section{Blogging activities}
After the edit of various Wikipedia voices, a lot about physics and mathematics, I started my blogging activities in January 2009 collaborating with the italian network Blogosfere for a popular science blog. This experience has led to the inauguration of the Italian language section of \href{http://researchblogging.org/}{Research Blogging}, a famous scientific content aggregator.\\
With the subsequent transformation of the network and the consequent closure of the blog, this experience was then transferred to two other blogs, a \href{http://dropseaofulaula.blogspot.it/}{personal blog in Italian} and \href{http://docmadhattan.fieldofscience.com/}{one in English} with the science blog network {\em Field of Science}. Especially among the topics covered I would point a series of posts within teaching: some examples are reported in the note\footnote{\href{http://dropseaofulaula.blogspot.it/2012/09/studio-in-classe-delle-biotracce-di-un.html}{Studio in classe delle biotracce di un pianeta extrasolare}, \href{http://dropseaofulaula.blogspot.it/2011/09/simulando-la-legge-di-hubble.html}{Simulando la legge di Hubble}}.

\section{Ph.D thesis}
In my Ph.D. thesis, {\em Rappresentazioni proiettive: Applicazioni nella teoria quantistica} ({\em Ray Representations: Application in quantum theory}), I have realized one unitary treatment of the ray representations theory, to develop applications in quantum theory.\\
In particular, I treated Bargmann's theory (1954), adding Grigore's (1996), Simms' (1968), Varadarajan's (1970) contributes about the determination of ray representations of a given Lie group, and Karpilovsky's (1994) contribute about the ray representation character.\\
These contributes, with Bargmann's study about abelian group and pseudo-orthogonal transformation group, are used in the study of rotation and Galilean groups.\\
In the first case, in according to Wigner (1939), Hurley (1966) and Altmann (1979), I present the emergence of semi-integer spin, so in the second case I study the Galilean group first in (3+1) dimensions (Bargmann, 1954), second in (2+1) dimensions (Grigore, 1995; Bose, 1995), and finally in (1+1) dimensions (Doebner and Mann, 1995).\\ 
In this last case, Doebner and Mann calculate the phase exponent of Galilean group with three parameters and determine the time depending infinitesimal generators. I extend the Doebner-Mann procedure to (2+1) and (3+1) dimensions, calculating a Galilean ray representation explicitly time depending.\\
The results of my work was presented ad 2007 Integrable Systems' Prague Conference, with the talk {\em Time dependent quantum generators for the Galilei group}. I submitted a paper with the same title to {\em Reports on Mathematical Physics}.\\
At the end of my Ph.D. work I studied the Wawrzycki's generalization (2004) of Bargmann's ray representations theory finding some points that need further deepening.
%
\section{Master Degree thesis}
In my Master Degree thesis, {\em Produzione esaltata di coppie $e^+ e^-$ nelle collisioni tra ioni pesanti relativistici}, I calculated the theoretical cross-section for the collision between heavy relativistic ions in order to explain the production of the couples $e^+ e^-$ using Preparata's theory (see G.Preparata, {\em QED Coherence in Matter}, World Scientific, 1995). I found that the results obteined according to Preparata's theory disagree with experimental data (see R.J.Porter, S.Beedoe, R.Bossingham et al. {\em Dielectron Cross Section Measurements in Nucleus-Nucleus Reactions at 1.0A GeV}, Phys. Rev. Lett. {\bf 79}, 1997).
\subsection*{Web pages about $\mu$ lepton}
Creation of web pages about the misuration of life time of $\mu$ lepton made during the course "Laboratorio di fisica delle particelle elementari" using Window's block notes. The site is accessible only in the \href{http://web.archive.org/web/20070207064033/http://www.fis.unical.it/gruppi/alteenergie/}{Wayback Machine version}.
%
\newpage
%
\section{Publications and working papers}
\begin{itemize}
	\item Ph.D thesis: {\em Rappresentazioni proiettive: Applicazioni alla teoria quantistica}, 2006, Universita' della Calabria
	\item Filippelli G. {\em Science Backstage} Scienza Aperta Vol. III (2012) - ComunicareFisica2010, Atti $3^\circ$ Convegno "Comunicare Fisica e altre Scienze", Frascati 12-16 aprile 2010, \href{http://www.lnf.infn.it/sis/frascatiseries/italiancollection/index_ita.php}{Frascati Physics Series - Italian Collection}
	\item Filippelli G., {\em Time dependent quantum generators for the Galilei group} Journal of Mathematical Physics, 52 (8) DOI: \href{http://dx.doi.org/10.1063/1.3621518}{10.1063/1.3621518}
	\item Filippelli G., Colaiacovo M., {\em United we stand. Siamo pronti per fare network?} Scienza Aperta Vol. IV (2014) - ComunicareFisica2012, Atti $4^\circ$ Convegno "Comunicare Fisica e altre Scienze", Torino 8-12 ottobre 2012, \href{http://www.lnf.infn.it/sis/frascatiseries/italiancollection/index_ita.php}{Frascati Physics Series - Italian Collection}
	\item {\em Moodle for Italian Astronomy Olimpiad} DOI: \href{http://dx.doi.org/10.5281/zenodo.6761}{10.5281/zenodo.6761}
\end{itemize}
%
\section{Conferences}
\begin{itemize}
	\item {\bf Torino 2016}: workshop Comunicare la Chimica nell'epoca del web 2.0: presentation \href{http://dropseaofulaula.blogspot.it/2016/09/web-20-da-fruitori-generatori-di.html}{Web 2.0: Da fruitori a generatori di informazione}
	\item {\bf Torino 2012}: ComunicareFisica: presentation of science blogging and divulgation activity
	\item {\bf Lucca 2012}: Lucca Comics\&Science: inaugural conference of the event 
	\item {\bf Frascati 2010}: ComunicareFisica: presentation of science blogging and divulgation activity
	\item {\bf Milano 2009}: Wordcamp: presentation of science blogging and divulgation activity
	\item {\bf Prague 2007}: 16th Integrable systems and quantum symmetries: presentation of the PhD work
	\item {\bf Bari 2004}: 12nd Workshop in Mechanical Statistics and Non Perturbative Fields Theory
	\item {\bf Cortona 2004}: 26th Informal Workshop in theoretical physics
\end{itemize}
\subsubsection*{Ph.D Schools}
\begin{itemize}
	\item {\bf Otranto 2005}: Scuola Nazionale di Otranto: XVIII Seminario Nazionale di Fisica Nucleare e Subnucleare
	\item {\bf Otranto 2003}: Scuola Nazionale di Otranto: XVI Seminario Nazionale di Fisica Nucleare e Subnucleare
\end{itemize}
%
%
%\section{Language knowledge}
%\begin{table}[h] %\centering
%\begin{tabular}{p{2cm}>{\bfseries}p{2.5cm}p{3.5cm}}
%& Italian  & alnguage mother\\
%& Inglese  & Ascolto: B1\\
%&  & Lettura: C1\\
%& & Interazione: B2\\
%& & Produzione orale: B1\\
%& & Scritto: B2\\
%\end{tabular}
%\end{table}
%
\vspace{2\baselineskip}
\noindent Milano, \today

\end{document}