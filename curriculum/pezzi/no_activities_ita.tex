\section{Informazioni personali}

\begin{flushleft}
	2 Febbraio, 1977, Cosenza, Italia\\
	Cittadinanza: italiana\\
	e-mail: gianluigi.filippelli@gmail.com\\
	%blog (it): \href{http://dropseaofulaula.blogspot.it/}{\em DropSea} | \href{http://sciencebackstage.blogspot.it/}{\em Science Backstage}\\
	%blog (en): \href{http://docmadhattan.fieldofscience.com/}{\em Doc Madhattan}\\
	%account twitter: \href{https://twitter.com/ulaulaman}{@ulaulaman}\\
	%\href{http://researchblogging.org/blogger/home/id/2181}{Research Blogging profile}\\
	%\href{https://it.wikipedia.org/wiki/Utente:Gianluigi}{Wikipedia user page}
\end{flushleft}

\section{In sintesi}

\begin{CV}
	
	\item[Titolo di studio:] Laurea (2001) e dottorato in fisica (2006); Master in e-learning (2011)
	\item[Esperienze editoriali:] Collaborazione con il \emph{network} Blogosfere (2009-2011); Collaborazione con il sito di critica fumettistica Lo Spazio Bianco (dal 2006); Collaborazione con il sito di divulgazione matematica \emph{Mathemathics in Europe} (in inglese); Caporedattore centrale per il\emph{ magazine online} EduINAF (dal 2017)
	\item[Esperienze didattiche:] Corso di esercitazione in analisi per ingegneri (2006-2007); Attivit\`a laboratoriali presso le scuole (2007); Insegnamento presso scuole superiori (2008-2017)
	\item[Programmazione:] Plugin per Wordpress (sul \href{https://profiles.wordpress.org/ulaulaman/}{repository ufficiale}; sul \href{https://github.com/ulaulaman}{repository GitHub})
	\item[Grafica e video:] Realizzazione ed editing di videointerviste per Lo Spazio Bianco; Rubrica \href{https://edu.inaf.it/category/rubriche/astrografiche/}{Astrografiche} su EduINAF; Editing e gestione canale YouTube EduINAF
	\item[Attivit\`a INAF:] Moodle Olimpiadi dell'Astronomia; Conferenze per le scuole; Progetto Cosmo Explorers; Partecipazione a festival e Notti dei ricercatori; Gestione e supervisione di attivit\`a di Alternanza Scuola-Lavoro e PCTO
\end{CV}

\section{Skills}

\begin{CV}
	
	\item[Sistemi operativi:] Dos, Windows, Linux (Ubuntu), BeOS
	\item[Software:] Word, Excel, Power Point, Star Office, Open Office, SalsaJ, Gimp, Shotcut (video-editing)
	\item[CMS:] Wordpress, Blogspot, Typepad, TiddlyWiki
	\item[E-learning:] eXe-Learning, Moodle, GeoGebra, software per LIM
	\item[Programmazione:] Pascal,Visual Basic 6, C\#, Sql, Scilab, Php, Javascript
	\item[Linguaggi di script:] Html, Css, \LaTeX, Markdown
	\item[Altro:] Esperienza in sistemi di acquisizione (VME e CAMAC, Field Point (National Instruments)) ed elaborazione dei dati
	
\end{CV}

\section{Esperienze professionali}
\subsection*{Attivit\`a didattiche e divulgative}

\begin{CV}
	\item[Mag 2017-present] Borsa di studio presso l'Osservatorio Astronomico di Brera per il rinnovo e la gestione del portale di didattica e divulgazione EduINAF, ora diventato magazine online.\\
	Compiti svolti: gestione tecnica del sito sviluppato con il CMS Wordpress con competenze di amministrazione (gestione del tema, del codice, delle utenze); gestione editoriale (revisione e programmazione dei contenuti; redazione diretta di contenuti; caporedattore centrale); gestione del canale YouTube (editing, caricamento e programmazione dei video); gestione del profilo twitter di EduINAF
	
	\item[Apr 2006-present] Redattore presso \href{http://www.lospaziobianco.it/}{\em Lo Spazio Bianco}, magazine on-line (CMS Wordpress - collaborazione a titolo gratuito)
	
	\item[Mar 2017-present] Collaborazione con il sito \href{http://mathematics-in-europe.eu/}{\em Mathematics in Europe} della European Mathematical Society (CMS Wordpress - collaborazione a titolo gratuito): \href{http://mathematics-in-europe.eu/?cat=154}{la rubrica aperiodica su MiE}
	
	\item[Lug 2014 (2 sett)] Scrittura di esercizi di fisica per Mondadori Education
	
	\item[Giu 2011-Giu 2012] Instructional designer\footnote{Figura del campo dell'e-learning con le competenze di progettazione e realizzazione di unit\`a didattiche e piattaforme VLE} per le Olimpiadi Italiane dell'Astronomia presso l'Osservatorio Astronomico di Brera (INAF), Milano
	
	\item[Ott 2010-Gen 2011] Stage in e-learning presso l'Osservatorio Astronomico di Brera (INAF), Milano
	
	\item[Gen 2009-Dic 2011] Collaborazione con il network {\em Blogosfere} per il blog di divulgazione scientifica \href{http://sciencebackstage.blogosfere.it/}{\em Science Backstage}\footnote{Link originale: \href{http://sciencebackstage.blogosfere.it/}{http://sciencebackstage.blogosfere.it/} Il blog \`e stato chiuso, ma \`e consultabile al link \href{https://web.archive.org/web/20111211024053/http://sciencebackstage.blogosfere.it/}{https://web.archive.org/web/20111211024053/http://sciencebackstage.blogosfere.it/}} (CMS Typepad)\\
	Altri blog scientifici: \href{http://dropseaofulaula.blogspot.com/}{\em DropSea} (it) | \href{http://docmadhattan.fieldofscience.com/}{\em Doc Madhattan} (en)
	
	\item[Apr-Mag 2007] Attivit\`a didattiche presso il Liceo Scientifico {\em G.B.Scorza}, Cosenza: presentazione di alcuni esperimenti e realizzazione di schede multimediali. Progetto "Lauree scientifiche" dell'INFM in collaborazione con l'Universit\`a della Calabria
	
	\item[2006-2007] Corso di esercitazione di Calcolo 2 (Analisi 2) per il Dipartimento di Ingegneria dell'Universit\`a della Calabria.
	
	\item[Mar 2004] Guida e performer per la mostra interattiva {\em Semplice e Complesso} realizzata dall'INFM presso l'Universit\`a della Calabria.
	
	\item[Mag 2002] Realizzazione di pagine web relative alla misura del tempo di vita del leptone $\mu$\footnote{Sito cancellato. Una sua versione archiviata \`e presente al link \href{http://web.archive.org/web/20070207064033/http://www.fis.unical.it/gruppi/alteenergie/}{http://web.archive.org/web/20070207064033/http://www.fis.unical.it/gruppi/alteenergie/}} per il gruppo di alte energie del Dipartimento di Fisica dell'Universit\`a della Calabria.
	
	\item[Link siti citati] \href{http://www.lospaziobianco.it/author/Gianluigi-Filippelli/}{http://www.lospaziobianco.it/author/Gianluigi-Filippelli/}\\
	\href{http://mathematics-in-europe.eu/?author=24}{http://mathematics-in-europe.eu/?author=24}\\
	\href{http://dropseaofulaula.blogspot.com/}{http://dropseaofulaula.blogspot.com/}\\
	\href{http://docmadhattan.fieldofscience.com/}{http://docmadhattan.fieldofscience.com/}
	
\end{CV}

\subsection*{Insegnamento nelle scuole}

\begin{CV}	
	\item[Feb-Mar 2017] Supplenza di matematica presso l'ITIS {\em Feltrinelli}, Milano
	
	\item[Gen-Feb 2017] Supplenza di matematica e fisica presso l'ITAS "G. Natta", Milano
	
	\item[Nov-Dic 2016] Supplenza di matematica e fisica presso il liceo scientifico {\em A. Volta}, Milano
	
	\item[Gen-Giu 2016] Docente di potenziamento di matematica presso l'IIS {\em G. Giorgi}, Milano
	
	\item[Ott 2014-Gen 2015] Supplenza di matematica presso l'IIS {\em G. Giorgi}, Milano
	
	\item[Feb 2014-Jun 2014] Supplenza di fisica presso l'IIS {\em F. Besta}, Milano
	
	\item[Ott 2013-Jan 2014] Supplenza di matematica presso il liceo scientifico {\em Claudio Cavalleri}, Parabiago (Milano)
	
	\item[Set-Ott 2013] Supplenza di matematica presso il liceo scientifico {\em Ettore Majorana}, Rho (Milano)
	
	\item[Feb-Lug 2013] Supplenza di matematica presso il liceo scientifico {\em Ettore Majorana}, Rho (Milano)
	
	\item[Nov 2012-Feb 2013] Supplenza di fisica presso l'IIS {\em G. Giorgi}, Milano
	
	\item[Set-Ott 2012] Supplenza di fisica presso il liceo scientifico {\em Claudio Cavalleri}, Parabiago (Milano) (due settimane)
	
	\item[Nov 2011-Giu 2012] Supplenza di fisica presso l'ITCS {\em Schiaparelli-Gramsci}, Milano
	
	\item[Nov 2011] Supplenza di fisica presso il liceo scientifico {\em Ettore Majorana}, Rho (Milano) (due settimane)
	
	\item[Nov-Dic 2010] Supplenza di fisica e matematica presso il liceo scientifico {\em Elio Vittorini}, Milano
	
	\item[Lug 2010] Corsi di recupero estivi in fisica presso l'ITIS {\em Galilei}, Milano
	
	\item[Mar-Giu 2010] Supplenza di matematica presso l'ITIS {\em Galilei}, Milano
	
	\item[Nov 2009] Supplenza di fisica presso l'ITIS {\em Galilei}, Milano (due settimane)
	
	\item[Lug 2009] Corsi di recupero estivi in matematica presso l'ITIS {\em ITIS Feltrinelli}, Milano
	
	\item[Nov 2008-Giu 2009] Supplenza di matematica presso l'ITIS {\em Feltrinelli}, Milano
	
	\item[Ott 2008] Supplenza di fisica presso il liceo linguistico {\em Levi}, Seregno (Milano) e l'ITIS {\em Mose' Bianchi}, Monza (Milano) (due settimane)
	
	\item[Gen-Giu 2008] Supplenza di fisica presso il liceo linguistico {\em Levi}, Seregno (Milano)
	
\end{CV}

\section{Curriculum Studiorum}

\begin{CV}
	\item[Gen 2010-Feb 2011] Master in e-learning con una tesi sulla progettazione e realizzazione di una piattaforma Moodle per lo studio della fisica e dell'astronomia
	
	\item[Nov-Dic 2008] Corso di programmazione: Visual Basics 6, C\#, SQL Server
	
	\item[12 Dic 2006] Ph.D. in fisica presso l'Universit\`a della Calabria, con la tesi {\em Rappresentazioni proiettive: Applicazioni nella teoria quantistica}, relatori Prof. G.Nistic\`o, Dott. D.Giuliano.
	
	\item[15 Mag 2002] Laurea in fisica presso l'Universit\`a della Calabria con la tesi {\em Produzione esaltata di coppie $e^+ e^-$ nelle collisioni tra ioni pesanti relativistici}, relatore Prof. R.Alzetta.
\end{CV}
%
\section{Conoscenze linguistiche}
Lingua madre: italiano
\subsection*{Autovalutazione: lingua inglese}
Ascolto: B1\\
Lettura: C1\\
Interazione: B2\\
Produzione orale: B1\\
Scritto: B2\\\\
%
\vspace{2\baselineskip}
\noindent Milano, \today