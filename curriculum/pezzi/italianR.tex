\section{Indirizzo}

\begin{flushleft}
  Via Paolo Lomazzo n.59,\\
  20154, Milano, Italy\\ 
  Cell: +39-3392315618\\
  e-mail: gianluigi.filippelli@gmail.com\\
\end{flushleft}

\section{Dettagli personali}
\begin{flushleft}
  Nato il: 2 Febbraio, 1977\\
  Nato a: Cosenza, Italia\\
  Cittadinanza: Italia
\end{flushleft}

\section{Attivit� didattica e Divulgazione scientifica}

\begin{CV}

\item[29/01/2009-] Inizio dell'attivit� di {\em blogging} per Science Backstage\\(http://sciencebackstage.blogosfere.it/)

\item[07/2009] Corso di recupero di Matematica per il 4.o superiore presso l'ITIS Feltrinelli di Milano

\item[11/2008-06/2009] Supplenza per il corso di Matematica per il trienno serale presso l'ITIS Feltrinelli di Milano

\item[10/2008] Supplenza per il corso di Scienze della Materia nel Liceo Linguistico Levi di Seregno (Milano). Supplenza per il corso di Fisica nell'Istituto Mos� Bianchi di Monza (Milano)

\item[01-06/2008] Supplenza per il corso di Scienze della Materia nel Liceo Linguistico Levi di Seregno (Milano)

\item[04-05/2007] Attivit� didattica presso il Liceo Scientifico G.B.Scorza, Cosenza: presentazione di alcune attivit� sperimentale e realizzazione di schede multimediali sulle esperienze

\item[A.A. 2006/2007] Corso di esercitazione di Calcolo 2 per Ingegneria presso l'Universit� della Calabria

\item[03/2004] Guida per la mostra Semplice e Complesso, tenutasi presso l'Universit� della Calabria

\item[05/2002] Creazione di pagine web sulla misurazione del tempo di vita media del leptone $\mu$\\(http://www.fis.unical.it/gruppi/alteenergie/)

\end{CV}

\section{Curriculum Studiorum}

\begin{CV}
\item[12/12/2006] Conseguimento del titolo di Dottore di Ricerca in Fisica con una tesi dal titolo {\em Rappresentazioni proiettive: Applicazioni nella teoria quantistica}, supervisori prof. G.Nistic�, dott. D.Giuliano

\item[2003--2006] Dottorato di ricerca in Fisica presso l'Universit� della Ca-labria

\item[15/05/2002] Laurea in Fisica presso l'Universit� della Calabria con una tesi dal titolo {\em Produzione esaltata di coppie $e^+ e^-$ nelle collisioni tra ioni pesanti relativistici}, relatore prof. R.Alzetta

\end{CV}

\section{Attivit� di ricerca}
\noindent {\bf Tesi di Dottorato}\\
Nel lavoro sviluppato per la tesi di Dottorato, {\em Rappresentazioni proiettive: Applicazioni nella teoria quantistica}, ho realizzato una trattazione unitaria ed autoconsistente della teoria delle rappresentazioni proiettive, per poterne successivamente svilupparne le applicazioni alla teoria quantistica.\\
In particolare alla teoria di Bargmann (1954) ho aggiunto i contributi di Grigore (1996), Simms (1968), Varadarajan (1970), per quel che riguarda la procedura di determinazione delle rappresentazioni proiettive di un dato gruppo di Lie G, e di Karpilovsky (1994) per il calcolo del carattere di una rappresentazione proiettiva.\\
Tali contributi, uniti allo studio di Bargmann (1954) sui gruppi abeliani e delle trasformazioni pseudo-ortogonali, saranno importanti nello studio dei gruppi di rilevanza fisica: gruppo delle rotazioni e gruppo di Galileo.\\
Nel primo caso realizzo una breve trattazione basandomi sui lavori di Wigner (1939), Hurley (1966), Altmann (1979), quindi nel caso del gruppo di Galileo, dopo aver esposto e semplificato lo studio di Bargmann (1954) in (3+1) dimensioni, realizzo una sintesi tra Grigore (1995) e Bose (1995) per quel che riguarda il gruppo di Galileo in (2+1) dimensioni, ottenendo una trattazione il pi� generale possibile del gruppo di Galileo in (2+1) dimensioni. Infine, si espone il contributo di Doebner e Mann (1995) allo studio del gruppo di Galileo in (1+1) dimensioni: i due studiosi includono le traslazioni temporali nel gruppo di simmetria, che diventa pertanto a tre parametri, ottenendo l'esponente di fase delle rappresentazioni proiettive e i generatori infinitesimi dipendenti dal tempo. In questo caso specifico estendo la procedura per determinare tali generatori alle dimensioni superiori, ricavando da questi anche una rappresentazione unitaria del gruppo di Galileo che risulta essere proiettiva.\\
Tale estensione � l'oggetto principale della presentazione, dal titolo {\em Time dependent quantum generators for the Galilei group}, presentata al congresso di Praga {\bf Integrable systems and quantum symmetries} 2007.\\
Infine mi sono interessato alla generalizzazione della teoria di Bargmann proposta da Wawrzycki (2004): in questo caso determino alcuni punti che necessitano di un'ulteriore approfondimento, per determinare la validit� o meno della proposta.\\\\
{\bf Tesi di Laurea}\\
Nella mia Tesi di Laurea, {\em Produzione esaltata di coppie $e^+ e^-$ nelle collisioni tra ioni pesanti relativistici}, ho calcolato la sezione d'urto teorica per la collisione tra ioni pesanti relativistici per spiegare la produzione delle coppie $e^+ e^-$, utilizzando la teoria di Preparata (vedi G.Preparata, {\em QED Coherence in Matter}, World Scientific, 1995). I risultati ottenuti a partire dalla teoria di Preparata sono in disaccordo con i risultati sperimentali (vedi R.J.Porter, S.Beedoe, R.Bossingham et al. {\em Dielectron Cross Section Measurements in Nucleus-Nucleus Reactions at 1.0A GeV}, Phys. Rev. Lett. {\bf 79}, 1997).
\newpage
{\bf Pubblicazioni}
 \begin{itemize}
  \item Tesi di Dottorato: {\em Rappresentazioni proiettive: Applicazioni alla teoria quantistica}, 2006, Universit� della Calabria
  %\item{\em Time dependent quantum generators for the Galilei group}, in fase di revisione
 \end{itemize}
{\bf Partecipazione a scuole e congressi}
 \begin{itemize}
  \item {\bf Milano 2009}: Wordcamp 2009
  \item {\bf Praga 2007}: XVI Integrable systems and quantum symmetries
  \item {\bf Otranto 2005}: Scuola Nazionale di Otranto: XVIII Seminario Nazionale di Fisica Nucleare e Subnucleare
  \item {\bf Bari 2004}: XII convegno di Meccanica Statistica e Teoria dei Campi non Perturbativa
  \item {\bf Cortona 2004}: Convegno Informale di Fisica Teorica - (XXVI edizione)
  \item {\bf Otranto 2003}: Scuola Nazionale di Otranto: XVI Seminario Nazionale di Fisica Nucleare e Subnucleare
 \end{itemize}

\section{Altre attivit�}

\begin{CV}

\item[17/11-22/12/2008] Corso di programmazione {\em Analisi e Sviluppo in ambiente Microsoft} (Visual Basic 6, C\#, SQL Server) presso la Esse.I, con periodo di affiancamento presso l'azienda privata Diagramma.

\end{CV}
%
%
%\section{Conoscenze linguistiche}
%\begin{table}[h] %\centering
%\begin{tabular}{p{2cm}>{\bfseries}p{2.5cm}p{3.5cm}}
%& Italiano  & lingua madre\\
%& Inglese  & Ascolto: B1\\
%&  & Lettura: C1\\
%& & Interazione: B2\\
%& & Produzione orale: B1\\
%& & Scritto: B2\\
%\end{tabular}
%\end{table}
%
\vspace{2\baselineskip}
\noindent Milano, \today