\ecvsection{Attivit� di ricerca}
 \ecvitem{Tesi di dottorato}{Nel lavoro sviluppato per la tesi di Dottorato, {\em Rappresentazioni proiettive: Applicazioni nella teoria quantistica}, ho realizzato una trattazione unitaria ed autoconsistente della teoria delle rappresentazioni proiettive, per poterne successivamente svilupparne le applicazioni alla teoria quantistica.}
 \ecvitem{}{In particolare alla teoria di Bargmann (1954) ho aggiunto i contributi di Grigore (1996), Simms (1968), Varadarajan (1970), per quel che riguarda la procedura di determinazione delle rappresentazioni proiettive di un dato gruppo di Lie G, e di Karpilovsky (1994) per il calcolo del carattere di una rappresentazione proiettiva.}
 \ecvitem{}{Tali contributi, uniti allo studio di Bargmann (1954) sui gruppi abeliani e delle trasformazioni pseudo-ortogonali, saranno importanti nello studio dei gruppi di rilevanza fisica: gruppo delle rotazioni e gruppo di Galileo.}
 \ecvitem{}{Nel primo caso realizzo una breve trattazione basandomi sui lavori di Wigner (1939), Hurley (1966), Altmann (1979), quindi nel caso del gruppo di Galileo, dopo aver esposto e semplificato lo studio di Bargmann (1954) in (3+1) dimensioni, realizzo una sintesi tra Grigore (1995) e Bose (1995) per quel che riguarda il gruppo di Galileo in (2+1) dimensioni, ottenendo una trattazione il pi� generale possibile del gruppo di Galileo in (2+1) dimensioni. Infine, si espone il contributo di Doebner e Mann (1995) allo studio del gruppo di Galileo in (1+1) dimensioni: i due studiosi includono le traslazioni temporali nel gruppo di simmetria, che diventa pertanto a tre parametri, ottenendo l'esponente di fase delle rappresentazioni proiettive e i generatori infinitesimi dipendenti dal tempo. In questo caso specifico estendo la procedura per determinare tali generatori alle dimensioni superiori, ricavando da questi anche una rappresentazione unitaria del gruppo di Galileo che risulta essere proiettiva.}
 \ecvitem{}{Tale estensione � l'oggetto principale della presentazione, dal titolo {\em Time dependent quantum generators for the Galilei group}, presentata al congresso di Praga {\bf Integrable systems and quantum symmetries} 2007.}
 \ecvitem{}{Infine mi sono interessato alla generalizzazione della teoria di Bargmann proposta da Wawrzycki (2004): in questo caso determino alcuni punti che necessitano di un'ulteriore approfondimento, per determinare la validit� o meno della proposta.}
 \ecvitem{Tesi di Laurea}{Nel lavoro sviluppato per la tesi di Laurea, {\em Produzione esaltata di coppie $e^+ e^-$ nelle collisioni tra ioni pesanti relativistici}, ho calcolato la sezione d'urto della collisione tra ioni pesanti relativistici, puntando l'attenzione sulla parte responsabile della produzione delle coppie elettrone-positrone.}