\section{Adresses}

\begin{flushleft}
  Via Paolo Lomazzo n.59,\\
  20154, Milano, Italy\\ 
  Cell: +39-3392315618\\
  e-mail: gianluigi.filippelli@gmail.com\\
  blog (it): \href{http://dropseaofulaula.blogspot.it/}{\em DropSea} | \href{http://sciencebackstage.blogspot.it/}{\em Science Backstage}\\
  blog (en): \href{http://docmadhattan.fieldofscience.com/}{\em Doc Madhattan}\\
  account twitter: \href{https://twitter.com/ulaulaman}{@ulaulaman}\\
  \href{http://researchblogging.org/blogger/home/id/2181}{Research Blogging profile}
\end{flushleft}

\section{Date and place of birth}
\begin{flushleft}
  2nd February, 1977, Cosenza, Italy\\
  Citizen: italian
\end{flushleft}

\section{Professional Experience}
\subsection*{Outreach activities}

\begin{CV}
\item[Jun 2011-Jun 2012] Instructional designer for Italian Astronomy Olimpiads at Osservatorio Astronomico di Brera (INAF), Milano

\item[Oct 2010-Jan 2011] Stage in e-learning at Osservatorio Astronomico di Brera (INAF), Milano

\item[Apr 2006-present] Redactor at \href{http://www.lospaziobianco.it/}{\em Lo Spazio Bianco}

\item[Jan 2009-Dec 2011] Collaboration withe the network {\em Blogosfere} for the popular science blog \href{http://sciencebackstage.blogosfere.it/}{\em Science Backstage} (\href{https://web.archive.org/web/20111211024053/http://sciencebackstage.blogosfere.it/}{Wayback Machine version} | \href{http://sciencebackstage.blogspot.it/}{New version})\\
Other science blogs: \href{http://dropseaofulaula.blogspot.com/}{\em DropSea} (it) | \href{http://docmadhattan.fieldofscience.com/}{\em Doc Madhattan} (en)

\item[Apr-May 2007] Didactical activity at Liceo Scientifico {\em G.B.Scorza}, Cosenza: presentation of some experimental activity and realization of multimedial cards.

\item[Mar 2004] Guide for the extension {\em Semplice e Complesso} at Universita' della Calabria.

\item[May 2002] Creation of \href{http://www.fis.unical.it/gruppi/alteenergie/}{web pages  about the misuration of life time of $\mu$ lepton} (\href{http://web.archive.org/web/20070207064033/http://www.fis.unical.it/gruppi/alteenergie/}{Wayback Machine version}).

\end{CV}

\subsection*{Teaching activities}

\begin{CV}
\item[Nov 2016-Dec 2016] Math theacher at the high school {\em Liceo scientifico A. Volta}, Milano

\item[Jan 2016-Jun 2016] Strengthening Math teacher at the high school {\em I.I.S. G. Giorgi}, Milano
	
\item[Jul 2014 (2 weeks)] Exercises for Physics text books: Mondadori Education

\item[Oct 2014-Jun 2015] Math theacher at the high school {\em I.I.S. G. Giorgi}, Milano

\item[Feb 2014-Jun 2014] Physics theacher at the high school {\em I.I.S. F. Besta}, Milano

\item[Oct2013-Jan 2014] Math theacher at the high school {\em Liceo Claudio Cavalleri}, Parabiago (Milano)

\item[Sep 2013-Oct 2013] Math theacher at the high school {\em Liceo Ettore Majorana}, Rho (Milano)

\item[Feb 2013-Jul 2013] Math theacher at the high school {\em Liceo Ettore Majorana}, Rho (Milano)

\item[Nov 2012-Feb 2013] Physics theacher at high school {\em I.I.S. G. Giorgi}, Milano

\item[Sep 2012-Oct 2012] Physics theacher at high school {\em Liceo Claudio Cavalleri}, Parabiago (Milano) (two week)

\item[Nov 2011-Jun 2012] Physics theacher at the high school {\em I.T.C.S. Schiaparelli-Gramsci}, Milano

\item[Nov 2011] Physics theacher at the high school {\em Liceo Ettore Majorana}, Rho (Milano) (two week)

\item[Nov-Dec 2010] Physics and Math Teacher at the high school {\em Liceo Elio Vittorini}, Milano

\item[Jul 2010] Summer Physics course at the high school {\em ITIS Galilei}, Milano

\item[Mar 2010-Jun 2010] Math teacher at the high school {\em ITIS Galilei}, Milano

\item[Nov 2009] Physics teacher at the high school {\em ITIS Galilei}, Milano (two week)

\item[Jul 2009] Summer Math course at the high school {\em ITIS Feltrinelli}, Milano (two week)

\item[Nov 2008-Jun 2009] Math teacher at the high school {\em ITIS Feltrinelli}, Milano

\item[Oct 2008] Physic's teacher at the high school {\em Liceo Linguistico Levi}, Seregno (Milano) (two week)

\item[Oct 2008] Physic's teacher at the high school {\em Mose' Bianchi}, Monza (Milano) (two week)

\item[Jan-Jun 2008] Physics teacher at the high school {\em Liceo Linguistico Levi}, Seregno (Milano)

\item[2006-2007] Exercitation course of Calcolo 2 ({\em Calculus}) for Engineering at Universita' della Calabria.

\end{CV}

\section{Skills}

Operating systems: Dos, Win 9x, ME, XP, Linux (Ubuntu), BeOS\\
Software: Word, Excel, Power Point, Star Office, Open Office\\
E-learning software: eXe-Learning, Moodle\\
Programming and script languages: Pascal, Html, \LaTeX, Visual Basic 6, C\#, Sql, Scilab\\
Others: Experience on acquiring systems (VME e CAMAC, Field Point (National Instruments)) and data elaboration.

\section{Curriculum Studiorum}

\begin{CV}
\item[Jan 2010-Feb 2011] Master in e-learning with thesis about application of a Moodle platform in the study of physics and astronomy

\item[Nov-Dec 2008] Programming course: Visual Basics 6, C\#, SQL Server

\item[12th Dec 2006] Ph.D. in Physics, with the thesis {\em Rappresentazioni proiettive: Applicazioni nella teoria quantistica}, tutors Prof. G.Nistico', Dott. D.Giuliano.

\item[2003-2006] Doctoral student in Physics at Universita' degli Studi della Calabria.

\item[15th May 2002] Master degree in Physics with the thesis {\em Produzione esaltata di coppie $e^+ e^-$ nelle collisioni tra ioni pesanti relativistici}, supervisor Prof. R.Alzetta.
\end{CV}

\section{Research activity}
\subsection*{Ph.D thesis}
In my Ph.D. thesis, {\em Rappresentazioni proiettive: Applicazioni nella teoria quantistica} ({\em Ray Representations: Application in quantum theory}), I have realized one unitary treatment of the ray representations theory, to develop applications in quantum theory.\\
In particular, I treated Bargmann's theory (1954), adding Grigore's (1996), Simms' (1968), Varadarajan's (1970) contributes about the determination of ray representations of a given Lie group, and Karpilovsky's (1994) contribute about the ray representation character.\\
These contributes, with Bargmann's study about abelian group and pseudo-orthogonal transformation group, are used in the study of rotation and Galilean groups.\\
In the first case, in according to Wigner (1939), Hurley (1966) and Altmann (1979), I present the emergence of semi-integer spin, so in the second case I study the Galilean group first in (3+1) dimensions (Bargmann, 1954), second in (2+1) dimensions (Grigore, 1995; Bose, 1995), and finally in (1+1) dimensions (Doebner and Mann, 1995).\\ 
In this last case, Doebner and Mann calculate the phase exponent of Galilean group with three parameters and determine the time depending infinitesimal generators. I extend the Doebner-Mann procedure to (2+1) and (3+1) dimensions, calculating a Galilean ray representation explicitly time depending.\\
The results of my work was presented ad 2007 Integrable Systems' Prague Conference, with the talk {\em Time dependent quantum generators for the Galilei group}. I submitted a paper with the same title to {\em Reports on Mathematical Physics}.\\
At the end of my Ph.D. work I studied the Wawrzycki's generalization (2004) of Bargmann's ray representations theory finding some points that need further deepening.\\\\
{\bf Master Degree thesis}\\
In my Master Degree thesis, {\em Produzione esaltata di coppie $e^+ e^-$ nelle collisioni tra ioni pesanti relativistici}, I calculated the theoretical cross-section for the collision between heavy relativistic ions in order to explain the production of the couples $e^+ e^-$ using Preparata's theory (see G.Preparata, {\em QED Coherence in Matter}, World Scientific, 1995). I found that the results obteined according to Preparata's theory disagree with experimental data (see R.J.Porter, S.Beedoe, R.Bossingham et al. {\em Dielectron Cross Section Measurements in Nucleus-Nucleus Reactions at 1.0A GeV}, Phys. Rev. Lett. {\bf 79}, 1997).\\\\
\subsection*{Publications and working papers}
 \begin{itemize}
  \item Ph.D thesis: {\em Rappresentazioni proiettive: Applicazioni alla teoria quantistica}, 2006, Universita' della Calabria
  \item Filippelli G. {\em Science Backstage} Scienza Aperta Vol. III (2012) - ComunicareFisica2010, Atti $3^\circ$ Convegno "Comunicare Fisica e altre Scienze", Frascati 12-16 aprile 2010, \href{http://www.lnf.infn.it/sis/frascatiseries/italiancollection/index_ita.php}{Frascati Physics Series - Italian Collection}
  \item Filippelli G., {\em Time dependent quantum generators for the Galilei group} Journal of Mathematical Physics, 52 (8) DOI: \href{http://dx.doi.org/10.1063/1.3621518}{10.1063/1.3621518}
  \item Filippelli G., Colaiacovo M., {\em United we stand. Siamo pronti per fare network?} Scienza Aperta Vol. IV (2014) - ComunicareFisica2012, Atti $4^\circ$ Convegno "Comunicare Fisica e altre Scienze", Torino 8-12 ottobre 2012, \href{http://www.lnf.infn.it/sis/frascatiseries/italiancollection/index_ita.php}{Frascati Physics Series - Italian Collection}
  \item {\em Moodle for Italian Astronomy Olimpiad} DOI: \href{http://dx.doi.org/10.5281/zenodo.6761}{10.5281/zenodo.6761}
  \end{itemize}
\subsection*{International Conferences}
 \begin{itemize}
  \item {\bf Torino 2016}: workshop Comunicare la Chimica nell'epoca del web 2.0: presentation \href{http://dropseaofulaula.blogspot.it/2016/09/web-20-da-fruitori-generatori-di.html}{Web 2.0: Da fruitori a generatori di informazione}
  \item {\bf Torino 2012}: ComunicareFisica: presentation of science blogging and divulgation activity
  \item {\bf Frascati 2010}: ComunicareFisica: presentation of science blogging and divulgation activity
  \item {\bf Milano 2009}: Wordcamp: presentation of science blogging and divulgation activity
  \item {\bf Prague 2007}: 16th Integrable systems and quantum symmetries: presentation of the PhD work
  \item {\bf Bari 2004}: 12nd Workshop in Mechanical Statistics and Non Perturbative Fields Theory
  \item {\bf Cortona 2004}: 26th Informal Workshop in theoretical physics
 \end{itemize}
\subsubsection*{Ph.D Schools}
 \begin{itemize}
  \item {\bf Otranto 2005}: Scuola Nazionale di Otranto: XVIII Seminario Nazionale di Fisica Nucleare e Subnucleare
  \item {\bf Otranto 2003}: Scuola Nazionale di Otranto: XVI Seminario Nazionale di Fisica Nucleare e Subnucleare
 \end{itemize}
%
%
%\section{Language knowledge}
%\begin{table}[h] %\centering
%\begin{tabular}{p{2cm}>{\bfseries}p{2.5cm}p{3.5cm}}
%& Italian  & alnguage mother\\
%& Inglese  & Ascolto: B1\\
%&  & Lettura: C1\\
%& & Interazione: B2\\
%& & Produzione orale: B1\\
%& & Scritto: B2\\
%\end{tabular}
%\end{table}
%
\vspace{2\baselineskip}
\noindent Milano, \today