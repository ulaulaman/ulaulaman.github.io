\ecvsection{Attivita' di ricerca}
 \ecvitem{Tesi di dottorato}{Nel lavoro sviluppato per la tesi di Dottorato, {\em Rappresentazioni proiettive: Applicazioni nella teoria quantistica}, ho realizzato una trattazione unitaria ed autoconsistente della teoria delle rappresentazioni proiettive, per poterne successivamente svilupparne le applicazioni alla teoria quantistica.}
 %\ecvitem{}{In particolare alla teoria di Bargmann (1954) ho aggiunto i contributi di Grigore (1996), Simms (1968), Varadarajan (1970), per quel che riguarda la procedura di determinazione delle rappresentazioni proiettive di un dato gruppo di Lie G, e di Karpilovsky (1994) per il calcolo del carattere di una rappresentazione proiettiva.}
 %\ecvitem{}{Tali risultati sono stati poi applicati ai gruppi delle rotazioni (Wigner, 1939; Hurley, 1966; Altmann, 1979) ed al gruppo di Galileo in (3+1) dimensioni (Bargmann, 1954), in (2+1) dimensioni (Grigore, 1995; Bose, 1995) ed in (1+1) dimensioni (Doebner e Mann, 1995). In quest'ultimo case estendo la procedura proposta per determinare i generatori infinitesimi dipendenti dal tempo anche alle dimensioni superiori, ricavando da questi una rappresentazione proiettiva del gruppo esplicitamente dipendente dal tempo.}
 %\ecvitem{}{Tale estensione � l'oggetto principale della presentazione, dal titolo {\em Time dependent quantum generators for the Galilei group}, presentata al congresso di Praga {\bf Integrable systems and quantum symmetries} 2007.}
 \ecvitem{}{Aggiunti alla teoria di Bargmann (1954) contributi piu' recenti, utilizzando l'approccio di Doebner e Mann (1995), si calcolano le rappresentazioni proiettive del gruppo di Galileo esplicitamente dipendenti dal tempo in (2+1) e (3+1) dimensioni.}
 \ecvitem{}{Infine mi sono interessato alla generalizzazione della teoria di Bargmann proposta da Wawrzycki (2004): in questo caso determino alcuni punti che necessitano di un'ulteriore approfondimento, per determinare la validita' o meno della proposta.}
 \ecvitem{Tesi di Laurea}{Nel lavoro sviluppato per la tesi di Laurea, {\em Produzione esaltata di coppie $e^+ e^-$ nelle collisioni tra ioni pesanti relativistici}, ho calcolato la sezione d'urto della collisione tra ioni pesanti relativistici, puntando l'attenzione sulla parte responsabile della produzione delle coppie elettrone-positrone.}